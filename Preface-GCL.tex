This is a log book for observers wanting to see some of the brighter
globular clusters. All of the globular clusters listed in this logbook
are expected to be visually observable with an 8-inch (200mm)
telescope, from reasonably dark (Bortle 4) skies. Most of them may be
easily seen in a 4'' (100mm) or a 6'' (150mm) telescope, but there are
occasional, more difficult objects.

This is a compilation of observation log forms for each of the objects
accompanied by useful information about the object, 3 star charts, and
an image from the Digitized Sky Surveys. It may gain more features as
time progresses.

In making this logbook, globular clusters that met the following
criteria were chosen from the SAC database
(\url{http://www.saguaroastro.org/content/downloads.htm}):
\begin{itemize}
\item Globular cluster is not a Messier object
\item Globular cluster belongs to the NGC / IC catalogs
\item Globular cluster is marked brighter than `pretty faint' in
Dreyer descriptions (with a few hand-picked exceptions)
\end{itemize}
Please note that some of the more fainter objects could have been
included by accident. If you found some particular object
exceptionally tough / invisible, please report them to me via email at
$\langle$akarshsimha@gmail.com$\rangle$.

Note that the \textbf{magnitudes for most of these globular clusters
are clearly wrong}. This is a known issue. This is because data is
unavailable in ready form at the moment in KStars. Until this is
fixed, please trust the Dreyer descriptions and other internet
resources to determine observability.

Also note that when blue DSS images were unavailable \emph{red POSS II
/ UKSTU plates} have been used instead. This is the case for a handful
number of objects. Unfortunately, the present system does not permit
this to be explicitly mentioned.

It will be very useful to be able to read Dreyer's descriptions, that
have been provided on the log
forms. \url{http://obs.nineplanets.org/ngc.html} is a good resource to
learn about them.

Many of these globular clusters may not be visible at your
latitude. The book is hemisphere-neutral, and just lists objects
irrespective of southern / northern declination. It is important to
note that objects low in the horizon are made substantially more
difficult by airmass.

The book's content and structure is inspired by the Bangalore
Astronomical Society's (\url{http://bas.org.in}) observer
certification programs. The idea for this particular logbook came from
Mr. Naveen Nanjundappa.

Hope you will enjoy observing these globular clusters!

\\

\hfill -- Akarsh Simha
