Some of the technical terms used in the compilation are explained
\emph{in brief} here. Many resources that offer more detailed
explanations and further information are available on the
internet. You could alternatively also use KStars' AstroInfo project,
accessible from the KStars Help Menu. See
\url{http://edu.kde.org/kstars} for more.

\begin{itemize}
\item \textbf{Right Ascension} and \textbf{Declination} together
  constitute the \textbf{Equatorial Geocentric Coordinates} used in
  astronomy. It is a \emph{coordinate system} used to designate
  positions in the sky.

  Just like the location of a point on the earth is specified by the
  latitude and longitude, the location of a point in the sky is
  specified using the Right Ascension (RA) and Declination
  (Dec). Usually, these are denoted by the symbols $\alpha$ and
  $\delta$.
  
  The declination is simply a projection of the earth's latitudes onto
  the sky. For example, the north celestial pole lies at a declination
  of $+90\circdegree$, and is in the direction vertically above when
  standing at the north pole of the earth, which has a latitude of
  $+90\circdegree$. Southern declinations are considered
  negative. Declination is usually measured in degrees.

  Unlike longitude, RA is measured in hours. Just like an arbitrary
  longitude is chosen to be zero degrees (namely the prime meridian),
  a point called the \emph{First point of Aries} (usually denoted
  $\gamma$) is chosen to be the zero for RA. 1 hour corresponds to 15
  degrees.

\item \textbf{Precession; Epoch; J2000.0:} The axis about which the
  earth rotates is not stationary. Just like a spinning top, the earth
  wobbles causing the axis itself to move. This wobbling of the axis
  of the earth is described by motions called \emph{precession} and
  \emph{nutation}. Precession is the dominant of the two. As a result
  of precession, the pole star of today, Polaris, will no longer be
  near the pole several centuries later.

  The earth's axis traces a circle in the sky over a period of 26000
  years. This might sound like a small effect over a couple years, but
  astronomical positions are measured with rather high
  precision. Thus, precession effects must be taken into account.

  Most catalogs of stars and deep-sky objects list the RA and Dec of
  objects, but the RA and Dec of these objects actually vary because
  of precession. To remedy this, the catalogs provide RA and Dec at a
  specific instant in time, called an \emph{epoch}. Once the RA and
  Dec are known at this epoch, the RA and Dec at any other time may be
  calculated.

  A very common epoch is \emph{J2000.0} which ocurred at the beginning
  of the year 2000. Most catalogs specify the RA and Dec at this
  instant of time. Already in the year 2013, we can see noticable
  differences in the current coordinates when compared to the catalog
  coordinates at 2000.0

\item \textbf{Units of Angular Measure} are important, because
  distances and sizes in the sky are measured as an angle subtended at
  the earth.

  For instance, the moon and the sun are both about $\frac{1}{2}
  \circdegree$ in (angular) diameter -- they subtend an angle of
  $\frac{1}{2}\circdegree$ at the center of the earth.

  The degree is the most common unit of angular measure. A degree is
  subdivided into $60$ arcminutes. Arcminute is often denoted with a
  small apostrophe-like marking: $1\circdegree = 60'$. An arcminute is
  further divided into $60$ arcseconds. An arcsecond is often denoted
  with a double apostrophe: $1' = 60''$. Thus $1\circdegree = 3600''$.

  The earth rotates through $360\circdegree$ about its axis in 24 hours of
  time. Thus every hour of time corresponds to $15\circdegree$ of rotation
  of the earth. Thus, often in astronomy, the \emph{hour} is used as a
  measure of angle, exactly equal to $15\circdegree$. The sky, as viewed
  from earth, actually goes back to the same position in about $23$
  hours and $56$ minutes, a duration known as the \emph{sidereal day},
  because the revolution of the earth adds to the rotation of the
  earth. However, when hour is used as a measure of angle, it is
  exactly equal to $15\circdegree$. 60 minutes (of time) comprise an
  hour, and 60 seconds (of time) comprise a minute.

  Angles are sometimes quoted as decimal values in degrees or hours
  (eg: $31.25\circdegree$). The same angle may be quoted as a combination
  of integer degrees, (arc)minutes and (arc)seconds (eg: $31\circdegree
  15' 0''$) or hours, minutes (of time) and seconds (of time).

  In this compilation, RA is usually specified in the
  hours-minutes-seconds system, whereas Declination is usually
  specified in the degrees-minutes-seconds system.

\item \textbf{Magnitude scale} is almost always used in astronomy to
  express the brightnesses of astronomical objects. It's a logarithmic
  scale of brightness, which means increments in magnitude actually
  amount to multiplicative factors in brightness. In particular, in
  the magnitude scale, a difference of 5 in magnitude corresponds to
  $100 \times$ in brightness. The other important thing to note -- the
  higher the magnitude of a star / object, the \emph{fainter} it is! A
  magnitude $6$ star is a $100 x$ fainter than a magnitude $1$ star.

  If two stars have magnitudes $m_1$ and $m_2$, the ratio of their
  brightnesses is given by

  \begin{equation}
    \frac{I_2}{I_1} = 10^{0.4(m_1 - m_2)}
  \end{equation}

  Even if the object is an extended object (unlike a star, which
  almost always appears like a point through telescopes), the
  magnitude includes all the ``light'' (flux) from the object, no
  matter what the size of the object is. For extended objects, a
  definition of \textbf{surface brightness} is more
  convenient. Surface brightness, often measured in ``magnitudes per
  square arcsecond'' is a measure of how bright an object's surface
  is. So a large object ``A'' with the same magnitude as a small
  object ``B'', will still have a much larger (i.e. fainter) surface
  brightness than object ``B''.

\end{itemize}
