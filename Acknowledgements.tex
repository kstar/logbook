\section*{The Bangalore Astronomical Society}
The author of this compilation would like to acknowledge the Bangalore
Astronomical Society (BAS) for the inspiration behind this idea. In
particular, the author would like to thank the council members of the
BAS during 2013.

\section*{The Digitized Sky Survey}
The images used in this compilation come from the Digitized Sky Survey
plates, in particular, those from the POSS-II and UKSTU surveys.

The Digitized Sky Survey was produced at the Space Telescope Science
Institute under U.S. Government grant NAG W-2166. The images of these
surveys are based on photographic data obtained using the Oschin
Schmidt Telescope on Palomar Mountain and the UK Schmidt
Telescope. The plates were processed into the present compressed
digital form with the permission of these institutions.

The Second Palomar Observatory Sky Survey (POSS-II) was made by the
California Institute of Technology with funds from the National
Science Foundation, the National Aeronautics and Space Administration,
the National Geographic Society, the Sloan Foundation, the Samuel
Oschin Foundation, and the Eastman Kodak Corporation. The Oschin
Schmidt Telescope is operated by the California Institute of
Technology and Palomar Observatory.

The UK Schmidt Telescope was operated by the Royal Observatory
Edinburgh, with funding from the UK Science and Engineering Research
Council (later the UK Particle Physics and Astronomy Research
Council), until 1988 June, and thereafter by the Anglo-Australian
Observatory. The blue plates of the southern Sky Atlas and its
Equatorial Extension (together known as the SERC-J), the near-IR
plates (SERC-I), as well as the Equatorial Red (ER), and the Second
Epoch [red] Survey (SES) were all taken with the UK Schmidt telescope
at the AAO.

The images themselves were downloaded from the Mikulski Archive for
Space Telescopes (MAST; \url{http://archive.stsci.edu/}).

The author graciously acknowledges the free availability of DSS
imagery for non-profit activities, and the excellent web service
provided by the MAST.

\section*{KStars and other tools}

The author wishes to thank, in particular, the developers of KStars,
(\url{http://edu.kde.org/kstars}) the software that made the rendition
of star maps used in this compilation possible and made available the
data used in this compilation.

KStars is an astronomy software licensed under the GNU General Public
License v2 (\url{https://www.gnu.org/licenses/gpl-2.0}). It qualifies
as free software.

The typesetting of the charts was done using \LaTeX.

