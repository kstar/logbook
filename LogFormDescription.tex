\section{Description of the form}
\begin{itemize}
\item \textbf{The title} carries the common name of the object (if
  any) and the primary catalog number
\item \textbf{The subtitle} specifies the \emph{type} of the object
  (eg: Planetary Nebula, Galaxy etc) and the constellation in which it
  lies.
\item \textbf{Icons indicating observability} are shown on the right
  of the page.

  Objects that are expected to be visible from dark sites with small
  binoculars (eg: $10 \times 50$) are indicated with a binocular
  icon.

  Objects that are expected to be visible from city sites with smaller
  telescopes(eg: $4'' \sim 6''$) are indicated with a city skyline
  icon, accompanied by a small telescope icon.

  If the object is also expected to be visible in binoculars ($10
  \times 50$) from city skies, a tiny version of the same binocular
  icon is displayed just above the telescope icon, next to the city
  skyline icon.
\item \textbf{The Data table} lists some useful data about the
  object. 

  The first two rows list the RA and Dec, first current and then
  J2000.0.

  The ``Size'' field lists the size of the object in
  arcminutes. Imagine fitting the object into a rectangle in the
  sky. The larger (usually first) dimension, called the \emph{major
    axis} specifies the length of the rectangle. The smaller dimension
  (\emph{minor axis}) specifies the breadth of the rectangle. For
  example, $8' \times 3'$ means that the object will roughly fit into
  a rectangle with a length of $8$ arcminutes and a breadth of $3$
  arcminutes in the sky.

  The ``Position Angle'' field specifies the orientation of the major
  axis of the object (the ``length'' of the rectangle mentioned
  above). It is measured in degrees, from North towards East. If it
  says $90\degree$, it usually is invalid / unknown.

  The ``Magnitude'' field specifies the magnitude of the object.

  The ``Other Designation'' field carries the alternate catalog
  designation of the object.

\item \textbf{The Sky Chart} shows a map of the sky around the
  object.

  North is upwards in the map.
  
  The circle in the center represents a circle of $1\degree$ diameter
  on the sky.

  The black dots are stars. The green / red symbol in the center of
  the $1\degree$ circle represents the object. An effort is made to
  represent the size of the object accurately.

  The lines connecting stars are constellation lines, and help you
  visualize the constellations. Note that these are not standard and
  may differ across star charts. Always look up the name / designation
  of the star (or the RA/Dec of the object) to match against other
  charts.

  The fainter jagged, but solid, lines are the boundaries of
  constellations as defined by the IAU.

  The broken / dashed lines running up-down are lines of constant
  right ascension, just like longitudes on a map of the earth.

  The broken / dashed lines running left-right are lines of constant
  declination, just like latitudes on a map of the earth. If you see a
  thick horizontal line that extends through to the ends of the map,
  that represents the celestial equator.

\item \textbf{A DSS Image} is provided to give you a rough idea of
  what the object looks like. The appearance through your equipment,
  of course, could be drastically different depending on its
  capabilities! The DSS Image is an actual photo of the object taken
  with a fairly large, professional astronomical telescope. It is
  usually good to get a rough idea of what features may be visible and
  what may not be. Of course, it takes practice to realize which
  features in a DSS image you may actually expect to see through your
  telescope!

  The dimensions of the region of the sky in the image (in arcminutes)
  are specified below the image (eg: $30' \times 15'$). The first
  dimension is the width.

  In the DSS images, north is upwards, as with the map.

\item \textbf{The Observation Log} is where you log your own
  observations.
\end{itemize}

\section{Using the form}

\subsection{Wide-field Charts}
To use these forms, you will need to have wide-field star charts
showing the constellations handy. Preferably the chart should have RA
and Declination markings.

If you do not have a star atlases, you may purchase several
commercially available star atlases, or print out the Free Mag 7 Star
Atlas hosted at
\url{http://www.cloudynights.com/item.php?item_id=1052}.

You may alternately also use computer software to obtain wide-field
views. There are several free, open-source options, the most
recommended for this purpose being Stellarium. Stellarium may be
obtained for a variety of operating systems at
\url{http://www.stellarium.org}. Other recommended options include
KStars -- \url{http://edu.kde.org/kstars} and SkyChart --
\url{http://www.ap-i.net/skychart/start}, which also run on a variety
of operating systems.

\subsection{Locating the Constellations, finding a reference star}
First, make sure you are aware of the cardinal directions around
you. 

In the northern hemisphere, an easy way to identify north is to look
for the Big Dipper, a famous asterism of 7 stars, that is part of the
constellation Ursa Major. If the Big Dipper is not visible, Cassiopeia
is a good alternative. The constellation has the shape of an M,
$\Sigma$, W or 3 depending on the orientation.

In the southern hemisphere, you may look for the Southern Cross (Crux)
to identify south.

Use your wide-field star atlas to identify the constellation patterns
in the sky.

Prominent patterns that are easy to identify are the Great Square of
Pegasus, Cassiopeia, Orion, the head of Taurus the bull, Auriga, the
Southern Cross, the Big Dipper, Corvus, Scorpius, the Teapot in
Sagittarius. Use these as landmarks to find your way around the sky.

Identify a bright star (the bigger the circles, the brighter the stars
they represent), which we will refer to as the \emph{reference star},
within the finder chart embedded in the log. Locate the star in your
wide-field charts, and thereby locate it on the sky.

\subsection{Finding the object}

Once you have located the reference star, recalling that the sky maps
have north on the top, orient the book correctly to map what you see
in the sky with the sky chart in the logbook.

Then, a variety of options are at your disposal. One is to try to find
the location of the object in the sky precisely, by using a bunch of
stars, and point the telescope / binoculars to that location. For
example, if you see on the chart that the object is exactly between
two stars, you could just point your telescope exactly to that
location on the sky, using the two stars for reference. Another
technique is \emph{star hopping} -- work a route from the reference
star to the object using various other stars as landmarks.

Many an internet resource can help explain these techniques better.

Finally, you may need to pan the telescope a bit, or move your
binoculars around a bit to actually locate the object.

If the object is rather faint, you may need to precisely zero in on it
by using the star field around the object. To see the star field
around the object, the easiest way is to use software. The DSS images
may occasionally help you in this regard.

\subsection{Observing the object}

\emph{Averted vision}, also known as \emph{peripheral vision} is an
important observing technique. Use internet resources to understand
and master this technique.

Note that the magnitude is not a true indicator of the brightness of
the object as seen with a telescope. A large object ``A'' with the
same magnitude as a fainter object ``B'', will appear much fainter
than ``B'' because the light is spread over a larger area.

In the description provided in the logging form, for some objects, you
may notice a number of abbreviations specified. These constitute J L E
Dreyer's description of the object. Look up the internet for the
abbreviations used in Dreyer's descriptions to get a feel for what the
object looks like. Note that J L E Dreyer had larger telescopes and
was observing from dark skies when making these descriptions. However,
the descriptions are more apt than magnitudes in determining how
bright an object is.

%% TODO: Add a section on logging observations, in general
