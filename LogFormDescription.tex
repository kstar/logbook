\begin{itemize}
\item \textbf{The title} carries the common name of the object (if
  any) and the primary catalog number
\item \textbf{The subtitle} specifies the \emph{type} of the object
  (eg: Planetary Nebula, Galaxy etc) and the constellation in which it
  lies.
\item \textbf{The Data table} lists some useful data about the
  object. 

  The first two rows list the RA and Dec, first current and then
  J2000.0.

  The ``Size'' field lists the size of the object in
  arcminutes. Imagine fitting the object into a rectangle in the
  sky. The larger (usually first) dimension, called the \emph{major
    axis} specifies the length of the rectangle. The smaller dimension
  (\emph{minor axis}) specifies the breadth of the rectangle. For
  example, $8' \times 3'$ means that the object will roughly fit into
  a rectangle with a length of $8$ arcminutes and a breadth of $3$
  arcminutes in the sky.

  The ``Position Angle'' field specifies the orientation of the major
  axis of the object (the ``length'' of the rectangle mentioned
  above). It is measured in degrees, from North towards East. If it
  says $90\degree$, it usually is invalid / unknown.

  The ``Magnitude'' field specifies the magnitude of the object.

  The ``Other Designation'' field carries the alternate catalog
  designation of the object.

\item \textbf{The Sky Chart} shows a map of the sky around the
  object.

  North is upwards in the map.
  
  The circle in the center represents a circle of $1\degree$ diameter
  on the sky.

  The black dots are stars. The green / red symbol in the center of
  the $1\degree$ circle represents the object. An effort is made to
  represent the size of the object accurately.

  The lines connecting stars are constellation lines, and help you
  visualize the constellations. Note that these are not standard and
  may differ across star charts. Always look up the name / designation
  of the star (or the RA/Dec of the object) to match against other
  charts.

  The fainter jagged, but solid, lines are the boundaries of
  constellations as defined by the IAU.

  The broken / dashed lines running up-down are lines of constant
  right ascension, just like longitudes on a map of the earth.

  The broken / dashed lines running left-right are lines of constant
  declination, just like latitudes on a map of the earth. If you see a
  thick horizontal line that extends through to the ends of the map,
  that represents the celestial equator.

\item \textbf{A DSS Image} is provided to give you a rough idea of
  what the object looks like. The appearance through your equipment,
  of course, could be drastically different depending on its
  capabilities! The DSS Image is an actual photo of the object taken
  with a fairly large, professional astronomical telescope. It is
  usually good to get a rough idea of what features may be visible and
  what may not be. Of course, it takes practice to realize which
  features in a DSS image you may actually expect to see through your
  telescope!

  The dimensions of the region of the sky in the image (in arcminutes)
  are specified below the image (eg: $30' \times 15'$). The first
  dimension is the width.

  In the DSS images, north is upwards, as with the map.

\item \textbf{The Observation Log} is where you log your own
  observations.
\end{itemize}
