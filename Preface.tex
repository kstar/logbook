This is a log book for amateur astronomers intending to observe the
Messier objects.

The book's content and structure is inspired by the Bangalore
Astronomical Society's (\url{http://bas.org.in}) Messier certification
program.

This is a compilation of observation log forms for each of the 110
Messier objects accompanied by useful information about the object, a
star chart, and an image from the Digitized Sky Surveys. It may gain
more features as time progresses

Note that the magnitudes displayed for the nebulae M 43, M 78, M 20, M
8 and M 17 are blue magnitudes (as opposed to visual magnitudes) from
Dr. Wolfgang Steinicke's revised NGC/IC catalog, so the objects may
appear brighter to the eye than the magnitudes actually indicate.

Also note that the DSS image for M 37 is broken. It looks like it is
broken in all three POSS2/UKSTU bands. The infrared images of M 42, M
43 and M 16 are more representative of what is seen through a
telescope, and hence were replaced although blue plates and red plates
are available. The blue plate for M 106 had a defect, so the red plate
was used instead.

Hope you have fun observing these objects!

\\

\hfill --Akarsh Simha
