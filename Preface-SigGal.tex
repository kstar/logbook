This is a log book for observers wanting to see some of the bright
galaxies in the sky. All of the galaxies listed in this logbook are
expected to be visually observable with a 6-inch (150mm) telescope,
from reasonably dark (Bortle 4) skies.

This is a compilation of observation log forms for each of the objects
accompanied by useful information about the object, 3 star charts, and
an image from the Digitized Sky Surveys. It may gain more features as
time progresses.

Galaxies were programmatically chosen from the SAC database
(\url{http://www.saguaroastro.org/content/downloads.htm}) by applying
the following filters:
\begin{itemize}
\item Galaxy not a Messier object
\item Galaxy is marked ``cB'' or brighter in Dreyer's descriptions
\item Galaxy is brighter than 11.0 mag (to avoid tiny, high-surface
  brightness galaxies)
\end{itemize}
Since there was no human intervention involved, there could be
errors. However, I checked that most of the galaxies I consider bright
and significant were on the list, and those that I consider even
somewhat faint were not.

Many of these galaxies may not be visible at your latitude. The book
is hemisphere-neutral, and just lists objects irrespective of southern
/ northern declination. It is important to note that objects low in
the horizon are made substantially more difficult by airmass.

The book's content and structure is inspired by the Bangalore
Astronomical Society's (\url{http://bas.org.in}) observer
certification programs. The idea for this particular logbook came from
Mr. Naveen Nanjundappa.

Hope you will enjoy observing these galaxies!

\\

\hfill --Akarsh Simha
