This is a log book for amateur astronomers intending to observe
Hickson's Compact Groups.

This is a compilation of observation log forms for each of the Hickson
Compact Groups, accompanied by useful information about the object, a
star chart, and an image from the Digitized Sky Surveys. It may gain
more features as time progresses

The description contains the parameter $z$, used as the measure of
redshift in the astronomy community. For $z$ in the range of 0 to 0.1,
it's a good approximation that an increase by 0.01 in $z$ corresponds
to a distance of 133 million light years. Thus, an object with $z =
0.01$ is about 133 million light years away, $z = 0.02$ is about 266
million light years away, and so on.

The gray circle indicates a field-of-view of 1 degree, while the gray
square box around the object indicates a field-of-view of 15
arcminutes, typical of most of the DSS imagery.

The data comes from the NASA HEASARC database\footnote{The HEASARC
data is accessible at \url{http://heasarc.gsfc.nasa.gov/}}, and has
been clarified to be usable for non-profit and/or educational
purposes.

Hope you have fun observing the Hickson Compact Groups.

\\

\hfill --Akarsh Simha
